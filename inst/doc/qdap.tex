\documentclass[a4paper]{article}

\usepackage[margin=2cm]{geometry}
\usepackage[round]{natbib}
\usepackage{url}

\newcommand{\acronym}[1]{\textsc{#1}}
\newcommand{\class}[1]{\mbox{\textsf{#1}}}
\newcommand{\code}[1]{\mbox{\texttt{#1}}}
\newcommand{\pkg}[1]{{\normalfont\fontseries{b}\selectfont #1}}
\newcommand{\proglang}[1]{\textsf{#1}}

\usepackage{Sweave}
\begin{document}
\input{qdap-concordance}


\title{Introduction to \pkg{qdap}: Quantitative Discourse Analysis Package}
\author{Tyler W. Rinker}
\maketitle


\section*{Introduction}
This vignette gives a short introduction to basic workflow and function usage
for \pkg{qdap}. \pkg{qdap} is an \proglang{R} package designed
to assist in quantitative discourse analysis. The package stands as a bridge
between qualitative transcripts of dialogue and statistical analysis and
visualization.

The qdap package automates many of the tasks associated with quantitative 
discourse analysis of transcripts containing discourse including frequency 
counts of sentence types, words, sentence, turns of talk, syllable counts and 
other assorted analysis tasks. The package provides parsing tools for preparing 
transcript data. Many functions enable the user to aggregate data by any number 
of grouping variables providing analysis and seamless integration with other R 
packages that undertake higher level analysis and visualization of text. This 
provides the user with a more efficient and targeted analysis.

\section*{Work Flow}
\section*{Reading in Data}

\begin{Schunk}
\begin{Sinput}
> doc1 <- system.file("extdata/trans1.docx", package = "qdap")
> dat1 <- read.transcript(doc1)